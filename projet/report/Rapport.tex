%%%%%%%%%%%%%%%%%%%%%%%%%%
%%% Most used packages %%%
%%%%%%%%%%%%%%%%%%%%%%%%%%
\documentclass[a4paper]{memoir}
\usepackage[utf8]{inputenc}
\usepackage[T1]{fontenc}
\usepackage{hyperref}
\usepackage{amsmath}
\usepackage{amssymb}
\usepackage{amsthm}
\usepackage{stmaryrd}
\usepackage{graphicx}
\usepackage{parskip}
\usepackage{pgf}
\usepackage{amsmath}
\usepackage{amssymb}
\usepackage{tikz}
\usepackage{color}
\usetikzlibrary{arrows, automata, positioning}
\usepackage{lstautogobble}
\usepackage[framemethod=tikz]{mdframed}

%%% Used language
\usepackage[english]{babel}

%%% Default margin
\usepackage[left=3cm,right=3cm,top=3cm,bottom=3cm]{geometry}

%%% Default indentation
\setlength{\parindent}{0cm}


%%%%%%%%%%%%%%%%%%
%%% Cover page %%%
%%%%%%%%%%%%%%%%%%
%%% Template link :
%%% http://mirror.jmu.edu/pub/CTAN/info/latex-samples/TitlePages/titlepages.pdf
%%% Many thanks to Peter Wilson for is work.
\newlength{\drop}
\newcommand*{\titleM}{\begingroup % Misericords, T&H p 153
\drop = 0.08\textheight
\centering
\vspace*{\drop}

%%% Document title
{\Huge\bfseries Projet : Un problème de tomographie discrète}\\[\baselineskip]

%%% Document sub-title, can be repeated
{\large\scshape MOGPL : Modélisation et Optimisation par les Graphes et la Programmation Linéaire}\\[\baselineskip]

%%% Center text
\begin{vplace}[0.7]
    \textit{}
\end{vplace}

%%% Authors
{\scshape Realisé par\\ BECIRSPAHIC Lucas\\ et\\ ADOUM Robert}\par

\vspace*{2\drop}
\endgroup}


%%%%%%%%%%%%%%%%%%%
%%% Page number %%%
%%%%%%%%%%%%%%%%%%%
\let\footruleskip\undefined
\usepackage{fancyhdr} 
\fancyhf{}
\cfoot{\thepage}
\pagestyle{fancy}
\renewcommand{\headrulewidth}{0pt}
\renewcommand{\footrulewidth}{0pt}


%%%%%%%%%%%%%%%%%%%%%%%%%%%
%%% Section name format %%%
%%%%%%%%%%%%%%%%%%%%%%%%%%%
\setcounter{secnumdepth}{50}
\setcounter{tocdepth}{50}
\renewcommand{\thesection}{}
\renewcommand{\thesubsection}{}
\renewcommand{\thesubsubsection}{\arabic{section}.\arabic{subsection}.\arabic{subsubsection}}


%%%%%%%%%%%%%%%%%%%%
%%% Environments %%%
%%%%%%%%%%%%%%%%%%%%
%%% Proof
\newenvironment{myproof}[1][\proofname]{\proof[#1]\mbox{}\\*}{\endproof}





\begin{document}
    %%%%%%%%%%%%%%%%%%
    %%% Cover page %%%
    %%%%%%%%%%%%%%%%%%
    \begin{center}
    \titleM 
    \end{center}
    \clearpage
    
    %%%%%%%%%%%%%%%
    %%% Summary %%%
    %%%%%%%%%%%%%%%
    \begin{center}
    \tableofcontents
    \end{center}
    
    %%%%%%%%%%%%%%%%%%
    %%% First page %%%
    %%%%%%%%%%%%%%%%%%
    \newpage
    
    \section{I. Raisonnement par programmation dynamique}
    \subsection{1 - Première étape}
    \textbf{Question 1:}\\\\ Si l’on a calculé tous les $T(j, l)$, pour savoir si il est possible de colorier la ligne $l_{i}$ entière avec la séquence entière  il suffit de de regarder $T(m-1, k)$, si ce dernier vaut vrai alors il est possible de colorier la ligne entière avec la séquence entière. Si il vaut faux alors ce n'est pas possible.\\\\
    \textbf{Question 2:}
    \begin{itemize}
		\item Cas $l = 0$, $j\in{\left\lbrace 0,...,m-1 \right\rbrace }$: Vrai
		\item Cas $l \geqslant 1$, $j < s_{l}-1$: Faux
		\item Cas $l \geqslant 1$, $j = s_{l}-1$:
		\begin{itemize}
			\item Si $l = 1$ alors Vrai 
			\item Si $l \neq 1$ alors Faux
		\end{itemize}
	\end{itemize}
 	
 	\textbf{Question 3:}\\\\
 	La relation de récurrence permettant de calculer $T(j,l)$ est la suivante:\\\\
 	$T(j, l) = T(j-(s_{l}+1),l-1)$\\\\
 	En effet si l'on se trouve à la case j et que l'on veut savoir si il est possible de colorier la sous séquence $(s_{1}, ..., s_{l})$ il faut pouvoir colorier $s_{l}$ case(s) et laisser une case de séparation entre les coloration de $s_{l-1} et s_{l}$, il faut donc regarder si l'on peut colorier la ligne de la case 0 à $j - s_{l} - 1$ avec la sous séquence $(s_{1}, ..., s_{l-1})$
   

        

\begin{tabular}{|c||c||c|}
\hline
instances & nbCases & time \\ 
\hline
0 & 20 & 0.00042200088501 \\ 
\hline
1 & 25 & 0.000617027282715 \\ 
\hline
2 & 400 & 0.117752075195 \\ 
\hline
3 & 481 & 0.0961720943451 \\ 
\hline
4 & 625 & 0.182909011841 \\ 
\hline
5 & 675 & 0.199213027954 \\ 
\hline
6 & 900 & 0.51091504097 \\ 
\hline
7 & 1054 & 0.300116062164 \\ 
\hline
8 & 1400 & 0.43498301506 \\ 
\hline
9 & 2500 & 5.42304491997 \\ 
\hline
10 & 9801 & 8.71296691895 \\ 
\hline
\end{tabular}


\begin{figure}[h]
  \centering
  \includegraphics[width=0.75\linewidth]{../images/dynamique_instance9.png}
  \caption{Grille de l'instance numéro 9}
  \label{fig:instance9}
\end{figure}

\textbf{Question 9} En appliquant notre programme sur l'instance 11, on observe qu'en dépit de la petite taille de l'instance notre algorithme ne colorie rien. En effet quand une case peut être colorié à la fois en blanc et en noir notre algorithme ne fais rien. Si la couleur d'une case ne peut être déterminée de manière exacte grace aux contraintes elle ne sera pas colorié. Ce qui explique pourquoi notre algorithme ne colorie pas correctement l'instance 11.
\\
Une solution à ce problème est d'implémenter un algorithme de backtracking qui une fois la coloration effectuée, observe toutes les cases non coloriées et leur affecte 0 et 1 arbirtrairement puis on relance coloration avec la nouvelle grille. On réitère jusqu'a optenir une grille complète (dans ce cas fin de l'algorithme) ou une grille insolvabe. Si le grille ne peut pas être résoulue , on retourne jusqu'a l'affectation la plus récente et on prend l'autre couleur. On notera que cette algorithme prend beaucoup plus de temps pour résoudre les grilles, une autre approche est d'utiliser la PLNE.


 	
 	\newpage
 	\section{II. La PLNE à la rescousse}
    \subsection{1 - Modélisation}
    \textbf{Question 10:}\\\\ - $x_{ij}$ vaut 1 si la case (i, j)est coloriée en noir et 0 si coloriée en noir.\\\\
    - $y^{t}_{ij}$ vaut 1 si le $t_{ieme}$ bloc de la ligne $l_{i}$ commence à la case (i, j) et 0 sinon.\\\\
    - $z^{t}_{ij}$ vaut 1 si le $t_{ieme}$ bloc de la colonne $c_{j}$ commence à la case (i, j) et 0 sinon.\\\\
    Par conséquent on a : $y^{t}_{ij} = 1 \Rightarrow\sum \limits_{{k=j}}^{j+s_{t}-1} x_{ik} = s_{t}$\\ Et donc $\sum \limits_{{k=j}}^{j+s_{t}-1} x_{ik} = y^{t}_{ij} \times s_{t} $ \\ Par conséquent la condition est: $\sum \limits_{{k=j}}^{j+s_{t}-1} x_{ik} \geq y^{t}_{ij} \times s_{t}  $ \\ Avec le même raisonnement on a pour les colonnes: $\sum \limits_{{k=i}}^{i+s_{t}-1} x_{kj} \geq z^{t}_{ij} \times s_{t} $\\
    \textbf{Question 11:}\\\\On a: $y^{t}_{ij} = 1 \Rightarrow\sum \limits_{{k=j}}^{j+s_{t}} y^{t+1}_{ik} = 0$\\ et $y^{t}_{ij} = 0 \Rightarrow\sum \limits_{{k=j}}^{j+s_{t}} y^{t+1}_{ik} \in \lbrace 0 , 1\rbrace$\\
    Et donc la condition est: $y^{t}_{ij} + \sum \limits_{{k=j}}^{j+s_{t}} y^{t+1}_{ik} \leq 1$\\
    Avec le même raisonnement on a pour les colonnes: $z^{t}_{ij} + \sum \limits_{{k=i}}^{i+s_{t}} z^{t+1}_{kj} \leq 1$\newpage
    \textbf{Question 12:}\\\\
   
  Min z = ?\\ 
  $$	
	s.c\left\{
    \begin{array}{ll}
         \sum \limits_{{k=j}}^{j+s_{t}-1} x_{ik} \geq y^{t}_{ij} \times s_{t}$ | $\forall i \in \lbrace 0, 1, 2,...,N-1\rbrace, \forall t \in \lbrace 1, 2,...,k_{i}\rbrace\\\\
         
        \sum \limits_{{k=i}}^{i+s_{t}-1} x_{kj} \geq z^{t}_{ij} \times s_{t} $ | $\forall j \in \lbrace 0, 1, 2,...,M-1\rbrace, \forall t \in \lbrace 1, 2,...,k_{j}\rbrace\\\\
        
        y^{t}_{ij} + \sum \limits_{{k=j}}^{j+s_{t}} y^{t+1}_{ik} \leq 1 $ | $\forall i \in \lbrace 0, 1, 2,...,N-1\rbrace, \forall t \in \lbrace 1, 2,...,k_{i}\rbrace\\\\
        
        z^{t}_{ij} + \sum \limits_{{k=i}}^{i+s_{t}} z^{t+1}_{kj} \leq 1$ | $\forall j \in \lbrace 0, 1, 2,...,M-1\rbrace, \forall t \in \lbrace 1, 2,...,k_{j}\rbrace\\\\
        
        \sum \limits_{{j=0}}^{M-1} y^{t}_{ij} = 1 $ | $\forall i \in \lbrace 0, 1, 2,...,N-1\rbrace, \forall t \in \lbrace 1, 2,...,k_{i}\rbrace\\\\
        
        \sum \limits_{{i=0}}^{N-1} z^{t}_{ij} = 1 $ | $\forall j \in \lbrace 0, 1, 2,...,M-1\rbrace, \forall t \in \lbrace 1, 2,...,k_{j}\rbrace\\\\
        
      x_{ij} \in \lbrace0,1\rbrace$ | $\forall i \in \lbrace 0, 1, 2,...,N-1\rbrace, \forall j \in \lbrace 0, 1, 2,...,M-1\rbrace\\\\
      
      y^{t}_{ij} \in \lbrace0,1\rbrace $ | $\forall i \in \lbrace 0, 1, 2,...,N-1\rbrace, \forall j \in \lbrace 0, 1, 2,...,M-1\rbrace, \forall t \in \lbrace 1, 2,...,k_{i}\rbrace\\\\
      
      z^{t}_{ij} \in \lbrace0,1\rbrace $ | $\forall i \in \lbrace 0, 1, 2,...,N-1\rbrace, \forall j \in \lbrace 0, 1, 2,...,M-1\rbrace, \forall t \in \lbrace 1, 2,...,k_{j}\rbrace\\\\
    \end{array}\\\\
\right.
$$
\\
\subsection{2 - Implantation et tests}
\textbf{Question 13: }\\\\
(N'oublions pas que j commence à 0 et termine à M-1)\\
- Pour une ligne $l_{i}$ le $l^{ieme}$ bloc ne peut commencer avant la case  $(i, \sum \limits_{{n=1}}^{l-1} (s_{n}+1))$ , ni commencer après la case $(i, M - s_{l} - \sum \limits_{{n = l+1}}^{k_{i}}(s_{n}+1))$ . 


     
\end{document}
