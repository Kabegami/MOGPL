%%%%%%%%%%%%%%%%%%%%%%%%%%
%%% Most used packages %%%
%%%%%%%%%%%%%%%%%%%%%%%%%%
\documentclass[a4paper]{memoir}
\usepackage[utf8]{inputenc}
\usepackage[T1]{fontenc}
\usepackage{hyperref}
\usepackage{amsmath}
\usepackage{amssymb}
\usepackage{amsthm}
\usepackage{stmaryrd}
\usepackage{graphicx}
\usepackage{parskip}
\usepackage{pgf}
\usepackage{amsmath}
\usepackage{amssymb}
\usepackage{tikz}
\usepackage{color}
\usetikzlibrary{arrows, automata, positioning}
\usepackage{lstautogobble}
\usepackage[framemethod=tikz]{mdframed}

%%% Used language
\usepackage[english]{babel}

%%% Default margin
\usepackage[left=3cm,right=3cm,top=3cm,bottom=3cm]{geometry}

%%% Default indentation
\setlength{\parindent}{0cm}


%%%%%%%%%%%%%%%%%%
%%% Cover page %%%
%%%%%%%%%%%%%%%%%%
%%% Template link :
%%% http://mirror.jmu.edu/pub/CTAN/info/latex-samples/TitlePages/titlepages.pdf
%%% Many thanks to Peter Wilson for is work.
\newlength{\drop}
\newcommand*{\titleM}{\begingroup % Misericords, T&H p 153
\drop = 0.08\textheight
\centering
\vspace*{\drop}

%%% Document title
{\Huge\bfseries Titre du document}\\[\baselineskip]

%%% Document sub-title, can be repeated
{\large\scshape Sous-titre du document}\\[\baselineskip]

%%% Center text
\begin{vplace}[0.7]
    \textit{Texte central}
\end{vplace}

%%% Authors
{\scshape Realisé par\\ Etudiant 1\\ et\\ Etudiant 2}\par

\vspace*{2\drop}
\endgroup}


%%%%%%%%%%%%%%%%%%%
%%% Page number %%%
%%%%%%%%%%%%%%%%%%%
\let\footruleskip\undefined
\usepackage{fancyhdr} 
\fancyhf{}
\cfoot{\thepage}
\pagestyle{fancy}
\renewcommand{\headrulewidth}{0pt}
\renewcommand{\footrulewidth}{0pt}


%%%%%%%%%%%%%%%%%%%%%%%%%%%
%%% Section name format %%%
%%%%%%%%%%%%%%%%%%%%%%%%%%%
\setcounter{secnumdepth}{50}
\setcounter{tocdepth}{50}
\renewcommand{\thesection}{}
\renewcommand{\thesubsection}{}
\renewcommand{\thesubsubsection}{\arabic{section}.\arabic{subsection}.\arabic{subsubsection}}


%%%%%%%%%%%%%%%%%%%%
%%% Environments %%%
%%%%%%%%%%%%%%%%%%%%
%%% Proof
\newenvironment{myproof}[1][\proofname]{\proof[#1]\mbox{}\\*}{\endproof}





\begin{document}
    %%%%%%%%%%%%%%%%%%
    %%% Cover page %%%
    %%%%%%%%%%%%%%%%%%
    \begin{center}
    \titleM 
    \end{center}
    \clearpage
    
    %%%%%%%%%%%%%%%
    %%% Summary %%%
    %%%%%%%%%%%%%%%
    \begin{center}
    \tableofcontents
    \end{center}
    
    %%%%%%%%%%%%%%%%%%
    %%% First page %%%
    %%%%%%%%%%%%%%%%%%
    \newpage
    
    \section{I. Raisonnement par programmation dynamique}
    \subsection{1 - Première étape}
    \textbf{Question 1:}\\\\ Si l’on a calculé tous les $T(j, l)$, pour savoir si il est possible de colorier la ligne $l_{i}$ entière avec la séquence entière  il suffit de de regarder $T(m-1, k)$, si ce dernier vaut vrai alors il est possible de colorier la ligne entière avec la séquence entière. Si il vaut faux alors ce n'est pas possible.\\\\
    \textbf{Question 2:}
    \begin{itemize}
		\item Cas $l = 0$, $j\in{\left\lbrace 0,...,m-1 \right\rbrace }$: Vrai
		\item Cas $l \geqslant 1$, $j < s_{l}-1$: Faux
		\item Cas $l \geqslant 1$, $j = s_{l}-1$:
		\begin{itemize}
			\item Si $l = 1$ alors Vrai 
			\item Si $l \neq 1$ alors Faux
		\end{itemize}
	\end{itemize}
 	
 	\textbf{Question 3:}\\\\
 	La relation de récurrence permettant de calculer $T(j,l)$ est la suivante:\\\\
 	$T(j, l) = T(j-(s_{l}+1),l-1)$\\\\
 	En effet si l'on se trouve à la case j et que l'on veut savoir si il est possible de colorier la sous séquence $(s_{1}, ..., s_{l})$ il faut pouvoir colorier $s_{l}$ case(s) et laisser une case de séparation entre les coloration de $s_{l-1} et s_{l}$, il faut donc regarder si l'on peut colorier la ligne de la case 0 à $j - s_{l} - 1$ avec la sous séquence $(s_{1}, ..., s_{l-1})$

\begin{tabular}{|c||c||c||c||c|}
\hline
instances & nb_cLines & time & nb_cCol & nbCases \\ 
\hline
0 & 6 & 0.0006148815155029297 & 7 & 20 \\ 
\hline
1 & 9 & 0.0023577213287353516 & 9 & 25 \\ 
\hline
2 & 74 & 0.24366474151611328 & 54 & 400 \\ 
\hline
3 & 39 & 0.16124892234802246 & 90 & 481 \\ 
\hline
4 & 112 & 0.321256160736084 & 112 & 625 \\ 
\hline
5 & 52 & 0.34999847412109375 & 61 & 675 \\ 
\hline
6 & 102 & 0.9388997554779053 & 100 & 900 \\ 
\hline
7 & 102 & 0.6064395904541016 & 76 & 1054 \\ 
\hline
8 & 115 & 0.7676417827606201 & 98 & 1400 \\ 
\hline
9 & 239 & 9.089713335037231 & 334 & 2500 \\ 
\hline
10 & 364 & 13.922056198120117 & 349 & 9801 \\ 
\hline
\end{tabular}





 	
 	\newpage
 	\section{II. La PLNE à la rescousse}
    \subsection{1 - Modélisation}
    \textbf{Question 10:}\\\\ - $x_{ij}$ vaut 1 si la case (i, j)est coloriée en noir et 0 si coloriée en noir.\\\\
    - $y^{t}_{ij}$ vaut 1 si le $t_{ieme}$ bloc de la ligne $l_{i}$ commence à la case (i, j) et 0 sinon.\\\\
    - $z^{t}_{ij}$ vaut 1 si le $t_{ieme}$ bloc de la colonne $c_{j}$ commence à la case (i, j) et 0 sinon.\\\\
    Par conséquent on a : $y^{t}_{ij} = 1 \Rightarrow\sum \limits_{{k=j}}^{j+s_{t}-1} x_{ik} = s_{t}$\\ Et donc $\sum \limits_{{k=j}}^{j+s_{t}-1} x_{ik} = y^{t}_{ij} \times s_{t} $ \\ Par conséquent la condition est: $\sum \limits_{{k=j}}^{j+s_{t}-1} x_{ik} \geq y^{t}_{ij} \times s_{t}  $ \\ Avec le même raisonnement on a pour les colonnes: $\sum \limits_{{k=i}}^{i+s_{t}-1} x_{kj} \geq z^{t}_{ij} \times s_{t} $\\
    \textbf{Question 11:}\\\\On a: $y^{t}_{ij} = 1 \Rightarrow\sum \limits_{{k=j}}^{j+s_{t}} y^{t+1}_{ik} = 0$\\ et $y^{t}_{ij} = 0 \Rightarrow\sum \limits_{{k=j}}^{j+s_{t}} y^{t+1}_{ik} \in \lbrace 0 , 1\rbrace$\\
    Et donc la condition est: $y^{t}_{ij} + \sum \limits_{{k=j}}^{j+s_{t}} y^{t+1}_{ik} \leq 1$\\
    Avec le même raisonnement on a pour les colonnes: $z^{t}_{ij} + \sum \limits_{{k=i}}^{i+s_{t}} z^{t+1}_{kj} \leq 1$\newpage
    \textbf{Question 12:}\\\\
   
  Min z = ?\\ 
  $$	
	s.c\left\{
    \begin{array}{ll}
         \sum \limits_{{k=j}}^{j+s_{t}-1} x_{ik} \geq y^{t}_{ij} \times s_{t}$ | $\forall i \in \lbrace 0, 1, 2,...,N-1\rbrace, \forall t \in \lbrace 1, 2,...,k_{i}\rbrace\\\\
         
        \sum \limits_{{k=i}}^{i+s_{t}-1} x_{kj} \geq z^{t}_{ij} \times s_{t} $ | $\forall j \in \lbrace 0, 1, 2,...,M-1\rbrace, \forall t \in \lbrace 1, 2,...,k_{j}\rbrace\\\\
        
        y^{t}_{ij} + \sum \limits_{{k=j}}^{j+s_{t}} y^{t+1}_{ik} \leq 1 $ | $\forall i \in \lbrace 0, 1, 2,...,N-1\rbrace, \forall t \in \lbrace 1, 2,...,k_{i}\rbrace\\\\
        
        z^{t}_{ij} + \sum \limits_{{k=i}}^{i+s_{t}} z^{t+1}_{kj} \leq 1$ | $\forall j \in \lbrace 0, 1, 2,...,M-1\rbrace, \forall t \in \lbrace 1, 2,...,k_{j}\rbrace\\\\
        
        \sum \limits_{{j=0}}^{M-1} y^{t}_{ij} = 1 $ | $\forall i \in \lbrace 0, 1, 2,...,N-1\rbrace, \forall t \in \lbrace 1, 2,...,k_{i}\rbrace\\\\
        
        \sum \limits_{{i=0}}^{N-1} z^{t}_{ij} = 1 $ | $\forall j \in \lbrace 0, 1, 2,...,M-1\rbrace, \forall t \in \lbrace 1, 2,...,k_{j}\rbrace\\\\
        
      x_{ij} \in \lbrace0,1\rbrace$ | $\forall i \in \lbrace 0, 1, 2,...,N-1\rbrace, \forall j \in \lbrace 0, 1, 2,...,M-1\rbrace\\\\
      
      y^{t}_{ij} \in \lbrace0,1\rbrace $ | $\forall i \in \lbrace 0, 1, 2,...,N-1\rbrace, \forall j \in \lbrace 0, 1, 2,...,M-1\rbrace, \forall t \in \lbrace 1, 2,...,k_{i}\rbrace\\\\
      
      z^{t}_{ij} \in \lbrace0,1\rbrace $ | $\forall i \in \lbrace 0, 1, 2,...,N-1\rbrace, \forall j \in \lbrace 0, 1, 2,...,M-1\rbrace, \forall t \in \lbrace 1, 2,...,k_{j}\rbrace\\\\
    \end{array}\\\\
\right.
$$
\\
\subsection{2 - Implantation et tests}
\textbf{Question 13: }\\\\
(N'oublions pas que j commence à 0 et termine à M-1)\\
- Pour une ligne $l_{i}$ le $l^{ieme}$ bloc ne peut commencer avant la case  $(i, \sum \limits_{{n=1}}^{l-1} (s_{n}+1))$ , ni commencer après la case $(i, M - s_{l} - \sum \limits_{{n = l+1}}^{k_{i}}(s_{n}+1))$ . 


     
\end{document}
